
% This LaTeX was auto-generated from an M-file by MATLAB.
% To make changes, update the M-file and republish this document.

\documentclass{article}
\usepackage{graphicx}
\usepackage{color}
\usepackage{listings}
\usepackage[framed]{mcode}
\usepackage{fullpage}
\usepackage{amsmath}
\usepackage[utf8x]{inputenc}
\usepackage{import}
\usepackage{setspace}
\usepackage{hyperref}
\definecolor{lightgray}{gray}{0.5}
\setlength{\parindent}{0pt}

\begin{document}

    
    
%\section*{}

 \title{BE 521: Homework 4 \\{\normalsize HFOs}
\\{\normalsize Spring 2019}} \author{58 points} \date{Due: Tuesday,
2/19/2019 11:59pm} \maketitle \textbf{Objective:} HFO detection and
cross-validation 
 \begin{center} \author{Andrew Clark \\
  \normalsize Collaborators: Vishal Tien, Joe Iwaysk \\}
\end{center}


\subsection*{HFO Dataset} High frequency oscillations (HFOs) are
quasi-periodic intracranial EEG transients with durations on the
order of tens of milliseconds and peak frequencies in the range of
80 to 500 Hz. There has been considerable interest among the
epilepsy research community in the potential of these signals as
biomarkers for epileptogenic networks.\\\\
In this homework exercise, you will explore a dataset of candidate
HFOs detected using the algorithm of Staba et al. (see article on
Canvas). The raw recordings from which this dataset arises come from
a human subject with mesial temporal lobe epilepsy and were
contributed by the laboratory of Dr. Greg Worrell at the Mayo Clinic
in Rochester, MN.\\\\
The dataset \verb|I521_A0004_D001| contains raw HFO clips that are
normalized to zero mean and unit standard deviation but are
otherwise unprocessed. The raw dataset contain two channels of data:
\verb|Test_raw_norm| and \verb|Train_raw_norm|, storing raw testing
and training sets of HFO clips respectively. The raw dataset also
contains two annotation layers: \verb|Testing windows| and
\verb|Training windows|, storing HFO clip start and stop times (in
microseconds) for each of the two channels above.
Annotations contain the classification by an ``expert'' reviewer
(i.e., a doctor) of each candidate HFO as either an HFO (2) or an
artifact (1). On ieeg.org and upon downloading the annotations,
You can view this in the "description" field. \\\\
After loading the dataset in to a \verb|session| variable as in
prior assignments you will want to familiarize yourself with the
\verb|IEEGAnnotationLayer| class. Use the provided "getAnnotations.m"
function to get all the annotations from a given dataset. The first
output will be an array of annotation objects, which you will see
also has multiple fields including a description field as well as start and stop times. Use
You can use the information outputted by getAnnotations to pull
each HFO clip.


\section{Simulating the Staba Detector (12 pts)} Candidate HFO clips
were detected with the Staba et al. algorithm and subsequently
validated by an expert as a true HFO or not. In this first section,
we will use the original iEEG clips containing HFOs and re-simulate
a portion of the Staba detection.
\begin{enumerate}
    \item How many samples exist for each class (HFO vs artifact) in
    the training set? (Show code to support your answer) (1 pt)

Start the session
\begin{lstlisting}
session = IEEGSession('I521_A0004_D001', 'andrewc', '/Users/andrewclark/Downloads/ieeg_password.bin' );
\end{lstlisting}

\color{lightgray} \begin{lstlisting}IEEGSETUP: Adding 'ieeg-matlab.jar' to dynamic classpath
Warning: Objects of edu/upenn/cis/db/mefview/services/TimeSeriesDetails class
exist - not clearing java 
Warning: Objects of edu/upenn/cis/db/mefview/services/TimeSeriesInterface class
exist - not clearing java 
IEEGSETUP: Found log4j on Java classpath.
URL: https://www.ieeg.org/services
Client user: andrewc
Client password: ****
\end{lstlisting} \color{black}
\begin{lstlisting}
%read in testing data
nr = ceil((session.data.rawChannels(1).get_tsdetails.getEndTime)/1e6*session.data.sampleRate);
test_data = session.data.getvalues(1:nr,1);
session.data;
\end{lstlisting}
\begin{lstlisting}
%read in training data
nr = ceil((session.data.rawChannels(2).get_tsdetails.getEndTime)/1e6*session.data.sampleRate);
train_data = session.data.getvalues(1:nr,2);
session.data;
\end{lstlisting}
\begin{lstlisting}
%Use get Annotations function

[allEvents, timesUSec, channels]=getAnnotations(session.data,'Training windows');
\end{lstlisting}
\begin{lstlisting}
%Pull all the descriptions from the output of getAnnotations
descriptions=zeros(1,length(allEvents));
for i=1:length(allEvents)
    d=allEvents(i).description;
    des=str2num(d);
    descriptions(i)=des;
    %descriptions(i)=allEvents(i).description
    %descriptions(i)=desc
end


%count the number of hfos and the number of artifacts
num_Hfos=0;
num_artifacts=0;

for j=1:length(descriptions)
    if descriptions(j)==1
        num_artifacts=num_artifacts+1;
    else
        num_Hfos=num_Hfos+1;
    end
end

num_Hfos
num_artifacts
\end{lstlisting}

\color{lightgray} \begin{lstlisting}
num_Hfos =

   101


num_artifacts =

    99

\end{lstlisting} \color{black}
\begin{lstlisting}
%As calculated by the code shown above, the number of HFO samples is 101
%and the number of artifact samples is 99.
\end{lstlisting}

    \item Using the training set, find the first occurrence of the
    first valid HFO and the first artifact.
        Using \verb|subplot| with 2 plots, plot the valid HFO's
        (left) and artifact's (right) waveforms. Since the units are
        normalized, there's no need for a y-axis, so remove it with
        the command \verb|set(gca,'YTick',[])|. (2 pts)

\begin{lstlisting}
%find first occurence of first valid HFO and first artifact

%sample rate
sample_rate=32556;

%First valid HFO
first_HFO=allEvents(1);
first_HFO
first_HFO_data=train_data(ceil(allEvents(1).start*1e-6*sample_rate):ceil(allEvents(1).stop*1e-6*sample_rate));%multiply by sample rate to get to sample indexing

%623
%First valid artifact
first_artifact=allEvents(2);
first_artifact
first_artifact_data=train_data(ceil(allEvents(2).start*1e-6*sample_rate):ceil(allEvents(2).stop*1e-6*sample_rate));%multiply by sample rate to get to sample indexing



%create time vectors


time_HFO=linspace(1/sample_rate,allEvents(1).stop*1e-6,length(first_HFO_data));


%time_HFO = linspace(1/sample_rate,(length(first_HFO_data)/sample_rate),length(first_HFO_data));
%time_first_artifact = linspace(1/sample_rate,(length(first_artifact_data)/sample_rate),length(first_artifact_data));
%time_first_art=linspace((1/allEvents(2).start),(length(allEvents(2).stop-allEvents(2).start)),length(first_artifact_data));
%time_1_art=allEvents(2).start/sample_rate:1/sample_rate:allEvents(2).stop/sample_rate;

time_1_art=linspace(allEvents(2).start*1e-6,allEvents(2).stop*1e-6,length(first_artifact_data));
\end{lstlisting}

\color{lightgray} \begin{lstlisting}
first_HFO = 

  <a href="matlab:help('IEEGAnnotation')">IEEGAnnotation</a>:

           type: 'trainWin'
    description: '2'
       isGlobal: True
    hasDuration: True
          start: 1
           stop: 16956
       channels: [1x1 IEEGTimeseries]

  <a href="matlab:methods(IEEGAnnotation)">Methods</a>, <a href="matlab:IEEGObject.openPortalSite()">main.ieeg.org</a>


first_artifact = 

  <a href="matlab:help('IEEGAnnotation')">IEEGAnnotation</a>:

           type: 'trainWin'
    description: '1'
       isGlobal: True
    hasDuration: True
          start: 16987
           stop: 36123
       channels: [1x1 IEEGTimeseries]

  <a href="matlab:methods(IEEGAnnotation)">Methods</a>, <a href="matlab:IEEGObject.openPortalSite()">main.ieeg.org</a>

\end{lstlisting} \color{black}
\begin{lstlisting}
%create plot
figure
subplot(1,2,1)
plot(time_HFO*1e3,first_HFO_data)
xlabel('Duration (Milliseconds)')
set(gca,'Ytick',[])
title('First Valid HFO')
subplot(1,2,2)
plot(time_1_art*1e3,first_artifact_data)
title('First Valid Artifact')
xlabel('Duration (Milliseconds)')
set(gca,'Ytick',[])
\end{lstlisting}


\includegraphics [width=5in]{clarkand_HW4_01.png}

    \item Using the \texttt{fdatool} in MATLAB, build an FIR
    bandpass filter of the equiripple type of order 100.
        Use the Staba et al. (2002) article to guide your choice of
        passband and stopband frequency. Once set, go to
        \texttt{File} \verb|->| \texttt{Export}, and export
        ``Coefficients'' as a MAT-File. Attach a screenshot of your
        filter's magnitude response. (Note: We will be flexible with
        the choice of frequency parameters within reason.) (3 pts)

\begin{lstlisting}
% \includegraphics[scale=0.3]{HW_screenshot_filter.png}\\
\end{lstlisting}

    \item Using the forward-backward filter function
    (\texttt{filtfilt}) and the numerator coefficients saved above,
        filter the valid HFO and artifact clips obtained earlier.
        You will need to make a decision about the input argument
        \texttt{a} in the \texttt{filtfilt} function. Plot these two
        filtered clips overlayed on their original signal in a two
        plot \texttt{subplot} as before. Remember to remove the
        y-axis. (3 pts)

use filt filt command
\begin{lstlisting}
filtered_HFO=filtfilt(coeff_1,1,first_HFO_data);
filtered_art=filtfilt(coeff_1,1,first_artifact_data);


figure
subplot(1,2,1)
plot(time_HFO*1e3,first_HFO_data)
hold on
plot(time_HFO*1e3,filtered_HFO)
xlabel('Duration (Milliseconds)')
set(gca,'Ytick',[])
title('First Valid HFO')
legend('Unfiltered', 'Filtered','Location','northwest')

subplot(1,2,2)
plot(time_1_art*1e3,first_artifact_data)
hold on
plot(time_1_art*1e3,filtered_art)
xlabel('Duration (Milliseconds)')
title('First Valid Artifact')
legend('Unfiltered', 'Filtered')
set(gca,'Ytick',[])
\end{lstlisting}


\includegraphics [width=5in]{clarkand_HW4_02.png}

    \item Speculate how processing the data using Staba's method may
    have erroneously led to a false HFO detection (3 pts)

Processing the data using Staba's method may erroneously lead to a false HFO detection because Staba's method invovles filtering the signal to remove frequency components above 500 Hz and below 80 Hz, which may make what is truly artifact appear to look like an HFO. In essence the bandpass filtering that is done in Saba's method will bring the two different types of waves (HFO and Artifacts) closer together in terms of frequency components, which may lead to a false HFO detection.

\end{enumerate}
\section{Defining Features for HFOs (9 pts)} In this section we will
be defining a feature space for the iEEG containing HFOs and
artifacts. These features will describe certain attributes about the
waveforms upon which a variety of classification tools will be
applied to better segregate HFOs and artifacts
\begin{enumerate}
    \item Create two new matrices, \verb|trainFeats| and
    \verb|testFeats|, such that the number of rows correspond to
    observations (i.e. number of training and testing clips)
        and the number of columns is two. Extract the line-length and
        area features (seen previously in lecture and Homework 3) from
        the normalized raw signals (note: use the raw signal from
        ieeg.org, do not filter the signal). Store the line-length value
        in the first column and area value for each sample in the second
        column of your features matrices. Make a scatter plot of the
        training data in the 2-dimensional feature space, coloring the
        valid detections blue and the artifacts red. (Note: Since we only
        want one value for each feature of each clip, you will
        effectively treat the entire clip as the one and only
        ``window''.) (4 pts)

\begin{lstlisting}
[allEvents_train, timesUSec_train, channels_train]=getAnnotations(session.data,'Training windows');
%run get annotations on the testing data
[allEvents_test, timesUSec_test, channels_test]=getAnnotations(session.data,'Testing windows');

%420 clips for testing
%200 clips for training

%create trainFeats matrix
trainFeats=zeros(length(allEvents_train),3);% first col is line-length, second is area, third is HFO or artifact characterization

%create testFeats matrix
testFeats=zeros(length(allEvents_test),3);

%define line length function
LLfn = @(x) sum(abs(diff(x)));

%define Area feature function
A = @(x) sum(abs(x));

%compute values for trainFeats matrix
for i=1:length(allEvents_train)
    trainFeats(i,1)=LLfn(train_data(ceil(allEvents_train(i).start*1e-6*sample_rate):ceil(allEvents_train(i).stop*1e-6*sample_rate)));
    trainFeats(i,2)=A(train_data(ceil(allEvents_train(i).start*1e-6*sample_rate):ceil(allEvents_train(i).stop*1e-6*sample_rate)));
    %characterize HFOs and Artifacts
    if str2num(allEvents_train(i).description) == 1
        trainFeats(i,3)=1; % 1 for artifact
    elseif str2num(allEvents_train(i).description) == 2
         trainFeats(i,3)=2; % 2 for HFO
    end


end


%compute values for testFeats matrix
for i=1:length(allEvents_test)
    testFeats(i,1)=LLfn(test_data(ceil(allEvents_test(i).start*1e-6*sample_rate):ceil(allEvents_test(i).stop*1e-6*sample_rate)));
    testFeats(i,2)=A(test_data(ceil(allEvents_test(i).start*1e-6*sample_rate):ceil(allEvents_test(i).stop*1e-6*sample_rate)));
    %characterize HFOs and Artifacts
    if str2num(allEvents_test(i).description) == 1
        testFeats(i,3)=1; % 1 for artifact
    elseif str2num(allEvents_test(i).description) == 2
        testFeats(i,3)=2; % 2 for HFO
    end
end

%create scatter plot of the training
figure
for i=1:length(trainFeats)
    if trainFeats(i,3) == 1
        plot(trainFeats(i,1), trainFeats(i,2),'.r','MarkerSize',10)
        hold on
    else
        plot(trainFeats(i,1),trainFeats(i,2),'.b','MarkerSize',10)
        hold on
    end
end
hold off
xlabel('Line-Length')
ylabel('Area')
title('Area v. Line-Length Scatter Plot')
legend('HFO','Artifact')
\end{lstlisting}


\includegraphics [width=5in]{clarkand_HW4_03.png}

    \item Feature normalization is often important. One simple
    normalization method is to subtract each feature by its mean and
    then divide by its standard deviation
        (creating features with zero mean and unit variance). Using
        the means and standard deviations calculated in your
        \emph{training} set features, normalize both the training
        and testing sets. You should use these normalized features for
        the remainder of the assignment.
    \begin{enumerate} \item What is the statistical term for the
    normalized value, which you have just computed? (1 pt)

\begin{lstlisting}
%take mean of each column, do element in column minus mean and divide by
%standard deviation

train_z_scores=zeros(length(trainFeats),2);

test_z_scores=zeros(length(testFeats),2);

%compute z scores for train data
for i=1:length(train_z_scores)
    train_z_scores(i,1)=(trainFeats(i,1)-mean(trainFeats(:,1)))/std(trainFeats(:,1)); %compute train line length z score
    train_z_scores(i,2)=(trainFeats(i,2)-mean(trainFeats(:,2)))/std(trainFeats(:,2)); %comput train are z score
end


%compute z scores for test data
for i=1:length(test_z_scores)
    test_z_scores(i,1)=(testFeats(i,1)-mean(trainFeats(:,1)))/std(trainFeats(:,1)); %compute test line length z score
    test_z_scores(i,2)=(testFeats(i,2)-mean(trainFeats(:,2)))/std(trainFeats(:,2)); %comput test area z score
end
\end{lstlisting}
The statistical term for the normalized value is the z-score.

	\item Explain why such a feature normalization might be critical
	to the performance of a $k$-NN classifier. (2 pts)

This feature normalization we have just completed will be critical to the performance of a KNN classifier because it ensures that all features used to train your KNN classifier on on the same scale. You want to ensure that changes in features used to train the KNN classifier are all equally weighted, this is important because the KNN classifier measures distances between points in the feature space to classify them. So, if these distances are on different scales then more weight would be erroneously given to features that are on bigger scales, even if they don't influence what the model is trying to predict even as much as features on smaller scales. This can be thought of as analagous to not adding two quantities together that have different units, if you did that your answer would make no physical sense.

\item Explain why (philosophically) you use the training feature means and
	standard deviations to normalize the testing set. (2 pts)

You use the training feature means and standard deviations to normalize the testing set because (philosophically) you do not have access to the testing dataset when you are creating a machine learning classifier, you only have access to the training dataset. For example if you are trying to actually predict a future outcome your testing dataset could be a dataset from the future (which doesn't exist yet and you don't have access to) so you need to train your model on the present data available. Thus, you would normalize the testing set to the training feature means and standard deviations, because those are the only means and standard deviations you have access to. Additionally, we can think of the classifier as classifying one testing observation at a time, so we can't normalize a single point and must use the training feature means and standard deviations.

    \end{enumerate}
\end{enumerate}
\section{Comparing Classifiers (20 pts)} In this section, you will
explore how well a few standard classifiers perform on this dataset.
Note, the logistic regression and $k$-NN classifiers are functions
built into some of Matlabs statistics packages. If you don't have
these (i.e., Matlab doesn't recognize the functions), we've provided
them, along with the LIBSVM mex files, in the \verb|lib.zip| file.
To use these, unzip the folder and add it to your Matlab path with
the command \texttt{addpath lib}. If any of these functions don't
work, please let us know.
\begin{enumerate}
 \item Using Matlab's logistic regression classifier function,
 (\texttt{mnrfit}), and its default parameters, train a model on the
 training set. Using Matlab's \texttt{mnrval} function, calculate
 the training error (as a percentage) on the data. For extracting
 labels from the matrix of class probabilities, you may find the
 command \texttt{[$\sim$,Ypred] = max(X,[],2)} useful\footnote{Note:
 some earlier versions of Matlab don't like the \texttt{$\sim$},
 which discards an argument, so just use something like
 \texttt{[trash,Ypred] = max(X,[],2)} instead.}, which gets the
 column-index of the maximum value in each row (i.e., the class with
 the highest predicted probability). (3 pts)

\begin{lstlisting}
%run mnr fit model
[B,dev,stats]=mnrfit((train_z_scores(:,1:2)),trainFeats(:,3))

pihat=mnrval(B,(train_z_scores(:,1:2)));

pihat_percentage=pihat*100;

[~, Ypred]=max(pihat_percentage, [], 2); %gets the column index of the maximumum of each row

%first column is 1 (artifact) second column is 2 (HFO)

%calculate training error

number_wrong_train=0;
for i=1:length(Ypred)
    if Ypred(i) ~= trainFeats(i,3)
        number_wrong_train=number_wrong_train+1;
    end

end

%calculate training error as a percent
training_error=(number_wrong_train/length(trainFeats))*100
\end{lstlisting}

\color{lightgray} \begin{lstlisting}
B =

    0.0499
   -2.4386
    0.8806


dev =

  144.2538


stats = 

  struct with fields:

         beta: [3�1 double]
          dfe: 197
         sfit: 1.2636
            s: 1
      estdisp: 0
         covb: [3�3 double]
    coeffcorr: [3�3 double]
           se: [3�1 double]
            t: [3�1 double]
            p: [3�1 double]
        resid: [200�2 double]
       residp: [200�2 double]
       residd: [200�1 double]


training_error =

   12.5000

\end{lstlisting} \color{black}
\begin{lstlisting}
% The training error is 12.5%.
\end{lstlisting}

 \item Using the model trained on the training data, predict the
 labels of the test samples and calculate the testing error. Is the
 testing error larger or smaller than the training error? Give one
 sentence explaining why this might be so. (2 pts)

Use the model trained on training data to predict labels of the test samples
\begin{lstlisting}
pihat_test=mnrval(B,(test_z_scores(:,1:2)));

pihat_percentage_test=pihat_test*100;

[~, Ypred_test]=max(pihat_percentage_test, [], 2);

%calculate the testing error
number_wrong_test=0;
for i=1:length(Ypred_test)
    if Ypred_test(i) ~= testFeats(i,3)
        number_wrong_test=number_wrong_test+1;
    end

end

%calculate training error as a percent
testing_error=(number_wrong_test/length(testFeats))*100
\end{lstlisting}

\color{lightgray} \begin{lstlisting}
testing_error =

   13.5714

\end{lstlisting} \color{black}
\begin{lstlisting}
%The testing error is 13.5714% while the training error is 12.5%. So, the
%testing error is slightly larger than the training error. The testing
%error is slightly larger because the model was trained on the training
%data, so inherently the model is biased towards the training data and better predicting the data it was trained on.
% When presented with a novel testing dataset, the model has to use the
% relationships it derived from the training set and apply it to the
% testing set. Since the testing set will not be identical to the training
% set, the testing error will be higher.
\end{lstlisting}

 \item
  \begin{enumerate}
   \item Use Matlab's $k$-nearest neighbors function,
   \texttt{fitcknn}, and its default parameters ($k$ = 1, among
   other things), calculate the training and testing errors. (3 pts)

\begin{lstlisting}
%create KNN model
Mdl=fitcknn((train_z_scores(:,1:2)),trainFeats(:,3),'NumNeighbors',1);
\end{lstlisting}
\begin{lstlisting}
%make predictions for training data

label_train=predict(Mdl,(train_z_scores(:,1:2)));


%calculate the training error
number_wrong_test=0;
for i=1:length(label_train)
    if label_train(i) ~= trainFeats(i,3)
        number_wrong_test=number_wrong_test+1;
    end

end

%calculate training error as a percent
training_error=(number_wrong_test/length(trainFeats))*100


label_test=predict(Mdl,(test_z_scores(:,1:2)));

%calculate the testing error
number_wrong_test=0;
for i=1:length(label_test)
    if label_test(i) ~= testFeats(i,3)
        number_wrong_test=number_wrong_test+1;
    end

end

%19

%calculate training error as a percent
testing_error=(number_wrong_test/length(testFeats))*100
\end{lstlisting}

\color{lightgray} \begin{lstlisting}
training_error =

     0


testing_error =

   17.3810

\end{lstlisting} \color{black}
\begin{lstlisting}
% The training error is 0%. The testing error is 17.3810%.
\end{lstlisting}

   \item Why is the training error zero? (2 pts)

The training error is zero because the number of neighbors (k) is specified to be one in the default parameters of the KNN classifer model. Since k=1, the KNN classifier is simply memorizing the training dataset and thus it will always accurately predicted the training data. Each datapoint in the training set will just be classified as the label it has, since it is its only and nearest neighbor.
\begin{lstlisting}
%
\end{lstlisting}

  \end{enumerate}
 \item In this question you will use the LIBSVM implementation of a
 support vector machine (SVM). LIBSVM is written in C, but we'll use
 the Matlab executable versions (with *.mex* file extensions). Type
 \texttt{svmtrain} and \texttt{svmpredict} to see how the functions
 are used\footnote{Matlab has its own analogous functions,
 \texttt{svmtrain} and \texttt{svmclassify}, so make sure that the
 LIBSVM files have been added to your path (and thus will superceed
 the default Matlab functions).}. Report the training and testing
 errors on an SVM model with default parameters. (3 pts)

Run svm model on train data
\begin{lstlisting}
svm_mdl=fitcsvm((train_z_scores(:,1:2)),trainFeats(:,3),'KernelFunction','RBF');


%generate training svm predictions
svm_predictions_train=predict(svm_mdl,(train_z_scores(:,1:2)));
\end{lstlisting}
\begin{lstlisting}
%calculate the training error
number_wrong=0;
for i=1:length(svm_predictions_train)
    if svm_predictions_train(i) ~= trainFeats(i,3)
        number_wrong=number_wrong+1;
    end

end

%calculate training error as a percent
training_error=(number_wrong/length(trainFeats))*100
\end{lstlisting}

\color{lightgray} \begin{lstlisting}
training_error =

    10

\end{lstlisting} \color{black}
\begin{lstlisting}
%generate svm testing predictions
svm_predictions_test=predict(svm_mdl,(test_z_scores(:,1:2)));

%calculate test error

%calculate the testing error
number_wrong=0;
for i=1:length(svm_predictions_test)
    if svm_predictions_test(i) ~= testFeats(i,3)
        number_wrong=number_wrong+1;
    end

end

%calculate training error as a percent
testing_error=(number_wrong/length(testFeats))*100
\end{lstlisting}

\color{lightgray} \begin{lstlisting}
testing_error =

   11.6667

\end{lstlisting} \color{black}
\begin{lstlisting}
% The SVM classifier training error is 10% and the SVM classifier testing
% error is 11.6667%.
\end{lstlisting}

 \item It is sometimes useful to visualize the decision boundary of
 a classifier. To do this, we'll plot the classifier's prediction
 value at every point in the ``decision'' space. Use the
 \texttt{meshgrid} function to generate points in the line-length
 and area 2D feature space and a scatter plot (with the \verb|'.'|
 point marker) to visualize the classifier decisions at each point
 (use yellow and cyan for your colors). In the same plot, show the
 training samples (plotted with the '*' marker to make them more
 visible) as well. As before use blue for the valid detections and
 red for the artifacts. Use ranges of the features that encompass
 all the training points and a density that yields that is
 sufficiently high to make the decision boundaries clear. Make such
 a plot for the logistic regression, $k$-NN, and SVM classifiers. (4
 pts)

\begin{lstlisting}
% generate points using meshgrid

x_values=-5:0.01:5; %specify x values

y_values=-5:0.01:5; %specify y values

[X,Y]=meshgrid(x_values,y_values);

% run meshgrid through the model, color based on the prediction

%run meshgrid through the mnr fit model
meshgrid_mat=[X(:),Y(:)];
pihat_mesh=mnrval(B,(meshgrid_mat));

pihat_percentage_mesh=pihat_mesh*100;

[~, Ypred_mnr]=max(pihat_percentage_mesh, [], 2); %gets the column index of the maximumum of each row

plot_train_z_scores_X=[];

plot_train_z_scores_Y=[];

plot_groups_train=[];

for i=1:length(train_z_scores)
    plot_train_z_scores_X(i)=train_z_scores(i,1);
    plot_train_z_scores_Y(i)=train_z_scores(i,2);
    plot_groups_train(i)=trainFeats(i,3);
end

%create plot
figure
gscatter(X(:),Y(:),Ypred_mnr,'yc')
hold on
gscatter(plot_train_z_scores_X(:),plot_train_z_scores_Y(:),plot_groups_train(:),'rb','*')
legend('Artifact Space','HFO Space','Artifact','HFO','Location','northeast')
xlabel('Line-Length')
ylabel('Area')
title('Decision Boundary Plot for Logistic Regression Model')
\end{lstlisting}


\includegraphics [width=5in]{clarkand_HW4_04.png}
\begin{lstlisting}
%create plot for KNN
label_train_mesh=predict(Mdl,meshgrid_mat);

figure
gscatter(X(:),Y(:),label_train_mesh,'yc')
hold on
gscatter(plot_train_z_scores_X(:),plot_train_z_scores_Y(:),plot_groups_train(:),'rb','*')
legend('Artifact Space','HFO Space','Artifact','HFO','Location','northeast')
xlabel('Line-Length')
ylabel('Area')
title('Decision Boundary Plot for KNN Model')
\end{lstlisting}


\includegraphics [width=5in]{clarkand_HW4_05.png}
\begin{lstlisting}
%create plot for SVM
svm_predictions_test_mesh=predict(svm_mdl,meshgrid_mat);

figure
gscatter(X(:),Y(:),svm_predictions_test_mesh,'yc')
hold on
gscatter(plot_train_z_scores_X(:),plot_train_z_scores_Y(:),plot_groups_train(:),'rb','*')
legend('Artifact Space','HFO Space','Artifact','HFO','Location','northeast')
xlabel('Line-Length')
ylabel('Area')
title('Decision Boundary Plot for SVM Model')
\end{lstlisting}


\includegraphics [width=5in]{clarkand_HW4_06.png}

 \item In a few sentences, report some observations about the three
 plots, especially similarities and differences between them. Which
 of these has overfit the data the most? Which has underfit the data
 the most? (3 pts)

These three plots differ greatly in their appearance.
\begin{lstlisting}
%The logistic regression plot is divided by the classifer into the HFO and Artifact
% space by a line which cleanly classifies HFOs and Artifacts into two planes distinct
% planes (artifact on the left HFO on the right). But, many points are incorrectly classified by the logistic regression. However, the KNN model
% has many distinct artifact spaces and one distinct HFO space. This is
% because KNN classifies points based on the points around the point in
% question. If the K in the KNN classifier was increased (up to a suitable number), more data points
% would be looked at and there would be less distinct artifact and HFO
% spaces. Since k=1 in our KNN model here, each point in the training set
% (shown here)
% will be correctly classified, however there is alot of overfitting going
% on. This is because overfitting is when the classifier is too specific to
% the training data, which means it will break up the 2-D feature space
% into too many distinct HFO and artifact subspaces. In contrast, the SVM
% plot has only one distinct HFO space and one distinct artifact space, and these two spaces capture the large majority of the correct training labels.
% This is a better model than the KNN one shown because there is less
% overfitting going on and the SVM model will generalize better to testing
% data as the model is more rigorous.
% So, the KNN model overfits the data the most and the logistic regression model underfits the data the most. The SVM model
% produces the least amount of overfitting and underfitting (it is the best
% model of the three computed).
\end{lstlisting}

\end{enumerate}
\section{Cross-Validation (17 pts)} In this section, you will
investigate the importance of cross-validation, which is essential
for choosing the tunable parameters of a model (as opposed to the
internal parameters the the classifier ``learns'' by itself on the
training data).
\begin{enumerate}
 \item Since you cannot do any validation on the testing set, you'll
 have to split up the training set. One way of doing this is to
 randomly split it into $k$ unique ``folds,'' with roughly the same
 number of samples ($n/k$ for $n$ total training samples) in each
 fold, training on $k-1$ of the folds and doing predictions on the
 remaining one.
 In this section, you will do 10-fold cross-validation, so create a
 cell array\footnote{A cell array is slightly different from a
 normal Matlab numeric array in that it can hold elements of
 variable size (and type), for example \texttt{folds\{1\} = [1 3 6]; folds\{2\}
 = [2 5]; folds\{3\} = [4 7];}.} \texttt{folds} that contains 10
 elements, each of which is itself an array of the indices of
 training samples in that fold. You may find the \texttt{randperm}
 function useful for this.
 Using the command \texttt{length(unique([folds\{:\}]))}, show that
 you have 200 unique sample indices (i.e. there are no repeats
 between folds). (2 pts)

Create a cell array folds that contains 10 elements, each of which itself is an array of the indices of training samples in that fold
\begin{lstlisting}
permutations=randperm(200,200);

%perm_reshape=reshape(permutations,[

folds=cell(10,1);

folds{1}=permutations(1:20);
folds{2}=permutations(21:40);
folds{3}=permutations(41:60);
folds{4}=permutations(61:80);
folds{5}=permutations(81:100);
folds{6}=permutations(101:120);
folds{7}=permutations(121:140);
folds{8}=permutations(141:160);
folds{9}=permutations(161:180);
folds{10}=permutations(181:200);

length(unique([folds{:}]))
\end{lstlisting}

\color{lightgray} \begin{lstlisting}
ans =

   200

\end{lstlisting} \color{black}

 \item Train a new $k$-NN model (still using the default parameters)
 on the folds you just created. Predict the labels for each fold
 using a classifier trained on all the other folds. After running
 through all the folds, you will have label predictions for each
 training sample.
  \begin{enumerate}
   \item Compute the error (called the validation error) of these
   predictions. (3 pts)

\begin{lstlisting}
%We need to iterate through the folds and train on all the folds that aren't the index we are on
shifted_num_array=folds;
%number_wrong=0;
error_list=[];
for i=1:length(folds)
   training_set=[shifted_num_array{1:9}];

    Mdl_folds=fitcknn(train_z_scores(training_set,1:2),trainFeats(training_set,3));
    predicted_fold=predict(Mdl_folds,(train_z_scores(shifted_num_array{10}(1:20),1:2)));
    compare=trainFeats(shifted_num_array{10}(1:20),3);
    error_list(i)=((sum(predicted_fold~=compare))/20)*100;

    shifted_num_array=circshift(shifted_num_array,1);
end

validation_error=mean(error_list)
\end{lstlisting}

\color{lightgray} \begin{lstlisting}
validation_error =

   19.5000

\end{lstlisting} \color{black}
\begin{lstlisting}
% The validation error for the folds computed on this iteration is 20.5%.
\end{lstlisting}

\item How does this error compare (lower,
   higher, the same?) to the error you found in question 3.3? Does
   it make sense? (2 pts)

\begin{lstlisting}
% The validation error is 20.5%, which is higher than the error found in
% 3.3 which was 17.3810%. This makes sense because you are only using 180
% datapoints to train your model with this folds method, as opposed to all
% 200 datapoints used to train the model in 3.3. A higher number of
% training points will lead to a lower training error because the
% classifier
% will have more data to compare against and can thus create a more
% rigorous model that will generalize better.
\end{lstlisting}

  \end{enumerate}
 \item Create a parameter space for your $k$-NN model by setting a
 vector of possible $k$ values from 1 to 30. For each values of $k$,
 calculate the validation error and average training error over the
 10 folds.
  \begin{enumerate}
   \item Plot the training and validation error values over the
   values of $k$, using the formatting string \texttt{'b-o'} for the
   validation error and \texttt{'r-o'} for the training error. (4
   pts)

\begin{lstlisting}
% %We need to iterate through the folds and train on all the folds that aren't the index we are on
shifted_num_array=folds;
number_wrong=0;
error_list=[];
train_list=[];
final_valid_error_list=[];
training_error_list=[];
for k=1:30
    for i=1:length(folds)
        training_set=[shifted_num_array{1:9}];

        Mdl_folds=fitcknn(train_z_scores(training_set,1:2),trainFeats(training_set,3),'NumNeighbors',k);
        predicted_fold_train=predict(Mdl_folds,(train_z_scores(training_set,1:2)));
        compare_train=(trainFeats(training_set,3));
        train_list(i)=((sum(predicted_fold_train~=compare_train))/180)*100;

        predicted_fold=predict(Mdl_folds,(train_z_scores(shifted_num_array{10}(1:20),1:2)));
        compare=trainFeats(shifted_num_array{10}(1:20),3);
        error_list(i)=((sum(predicted_fold~=compare))/20)*100;

        shifted_num_array=circshift(shifted_num_array,1);
    end
    final_valid_error_list(k)=mean(error_list);
    train_error_list(k)=mean(train_list);
end
\end{lstlisting}
\begin{lstlisting}
%create plot

k=1:30;

figure
plot(k,final_valid_error_list,'b-o')
hold on
plot(k,train_error_list,'r-o')
xlabel('Value of k')
ylabel('Error (%)')
title('KNN Model Training and Validation Errors Over k=1 to k=30')
legend('Validation Error','Training Error')
\end{lstlisting}


\includegraphics [width=5in]{clarkand_HW4_07.png}

\item What is the optimal $k$ value and its error? (1 pts)

\begin{lstlisting}
optimal_valid_error=final_valid_error_list(30)
optimal_train_error=train_error_list(30)
\end{lstlisting}

\color{lightgray} \begin{lstlisting}
optimal_valid_error =

   12.5000


optimal_train_error =

   12.2778

\end{lstlisting} \color{black}
The optimal k is k=30. This value was chosen because this value of k gave the lowest validation error when many different sets of permutations were computed. It is possible to get a lower validation error at a lower value of k (I have observed somewhere in the range between k=8 and k =15), but the variation in the value of k makes a choice in this range suboptimal. For example, if you pick k=10 as the optimal value based on k=10 giving the lowest validation error for a certain set of folds, you may actually get a validation error higher than k=30 when the folds are computed again. Therefore, since the difference between the minmimum validation error and the validation error produced by k=30 has been observed to be quite small (approximately less than 3\%), k=30 is the optimal value as you can be most sure it will consistently give a validation within approximately 3\% of the lowest validation error possible with this model. The validation error for k=30 (with the current folds computed) is 12.5\% and the training error for k=30 is 12.28\%. It should be noted that, in general, as k increases error does not decrease but since we are only going up to k=30 the argument presented is valid.

   \item Explain why $k$-NN generally overfits less with higher
   values of $k$. (2 pts)

KNN generally overfits less with higher values of k because as k increases the KNN classifier considers more and more datapoints when it makes a decision on any the classification of any one datapoint. As such, the feature space it is classifying will be broken up into fewer distinct classification spaces so the training data will be "memorized" less and general patterns of datapoints will be deduced more often. We see that when k=1 the KNN only considers the point it is classifying at the time when it makes a classification decision, leading to overfitting as the feature space is paritioned too many times. So, as k increases more information will be taken into account and the general trends inherent in the data will be better incapsulated, creating a more rigorous model. It is important to note that the model does have not decreased error as k continues to increase (there exists an optimal value for k).

  \end{enumerate}
 \item
  \begin{enumerate}
   \item Using your optimal value for $k$ from CV, calculate the
   $k$-NN model's \textit{testing} error. (1 pts)

calculate testing error
\begin{lstlisting}
%train on the whole training set with k=30
optimal_k_model=fitcknn(train_z_scores(:,1:2),trainFeats(:,3),'NumNeighbors',30);

%predict on the test data
predicted_optimal_k=predict(optimal_k_model,(test_z_scores(:,1:2)));



%calculate the testing error
number_wrong_test=0;
for i=1:length(predicted_optimal_k)
    if predicted_optimal_k(i) ~= testFeats(i,3)
        number_wrong_test=number_wrong_test+1;
    end

end



%calculate testing error as a percent
testing_error_optimal=(number_wrong_test/length(testFeats))*100
\end{lstlisting}

\color{lightgray} \begin{lstlisting}
testing_error_optimal =

   11.4286

\end{lstlisting} \color{black}
\begin{lstlisting}
% The testing error with the optimal k (k=30) was calculated to be 11.4286%
\end{lstlisting}

\item How does this
   model's testing error compare to the $k$-NN model
   you trained in question 3.3? Is it the best of the three models
   you trained in Section 3? (2 pts)

This model's testing error is lower than the model trained in question 3.3, which had a testing error of 17.3810\%. So, this model is the best of the three models trained in Section 3.3 (in 3.3 SVM had a testing error of 11.667\% and logistic regression had a testing error of 13.5714\%, so this model's error is slightly lower than these 3 other models).

  \end{enumerate}
\end{enumerate}




\end{document}
    
