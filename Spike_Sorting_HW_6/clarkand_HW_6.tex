
% This LaTeX was auto-generated from an M-file by MATLAB.
% To make changes, update the M-file and republish this document.

\documentclass{article}
\usepackage{graphicx}
\usepackage{color}
\usepackage{listings}
\usepackage[framed]{mcode}
\usepackage{fullpage}
\usepackage{amsmath}
\usepackage[utf8x]{inputenc}
\usepackage{import}
\usepackage{setspace}
\usepackage{hyperref}
\definecolor{lightgray}{gray}{0.5}
\setlength{\parindent}{0pt}

\begin{document}

    
    
%\section*{}


\title{BE 521: Homework 6 \\{\normalsize Spike sorting}\\{\normalsize Spring 2019}}
\author{59 points}
\date{Due: Thursday, 03/5/2020 11:59pm}
\maketitle \textbf{Objective:} Detect and cluster spikes


\begin{center} \author{Andrew Clark \\
  \normalsize Collaborators: Alex Silva, Vishal Tien \\}
\end{center}


\subsection*{Overview}
In this homework, you will do some basic spike sorting using two different datasets. The first (\verb|I521_A0006_D001|) is from a crayfish neuromuscular junction, a good model for human central nervous system synapses\footnote{The sampling rate of this data is 2000 Hz, which is adequate for this homework's instructional purposes but usually inadequate for real spike sorting, which often uses sampling frequencies on the order of 20 kHz.}. Specifically, the data contains two simultaneous recordings: an extracellular recording from the third nerve (channel \verb|nerve|) of a crayfish abdominal ganglion, which contains six spontaneously active motor neurons, and an intracellular recording from the superficial flexor muscle (channel \verb|muscle|) innervated by this nerve. You will attempt to discern relationships between the classes of spike waveforms you extract from the motor nerve trace and elicited potentials seen in the muscle fiber recording.
Then, you will revisit a human intracranial EEG recording (\verb|I521_A0006_D002|) and use some of the techniques you've learned in class to build a more automated spike sorter.
Note: While spikes may have positive and negative deflections, we will only focus on positive spikes on this homework for simplicity.
\section{Spike Detection and Clustering (38 pts)}
In this section, you will explore some basic filtering and spike thresholding to ultimately compare spike clusters you pick out by eye to those selected by an automated algorithm.
\begin{enumerate}
    \item You can assume that the nerve samples have already been low-pass filtered. Here you will high-pass filter in order to remove signals like slow local field potentials and 60 Hz power line noise. Create a 4th order \textit{elliptic filter} with 0.1 dB of ripple in the passband, a stopband 40 dB lower than the peak value in the passband, and a passband edge frequency of 300 Hz (see Matlab's \verb|ellip| function and make sure your give the edge frequency in the correct normalized form). The statement to create this filter (defined by the filter coefficients \verb|b| and \verb|a|) should look something like
  \begin{lstlisting}
	[b,a]=ellip(n,Rp,Rs,Wp,'high')
  \end{lstlisting}
  Clearly specify the denominator and numerator coefficients obtained for your filter function. (2pts)

\begin{lstlisting}
session = IEEGSession('I521_A0006_D001', 'andrewc', '/Users/andrewclark/Downloads/ieeg_password.bin' );
\end{lstlisting}

\color{lightgray} \begin{lstlisting}IEEGSETUP: Adding 'ieeg-matlab.jar' to dynamic classpath
IEEGSETUP: Found log4j on Java classpath.
URL: https://www.ieeg.org/services
Client user: andrewc
Client password: ****
\end{lstlisting} \color{black}
\begin{lstlisting}
session.data


nr = ceil((session.data.rawChannels(1).get_tsdetails.getEndTime)/1e6*session.data.sampleRate);
allData_muscle = session.data.getvalues(1:nr,1);
allData_nerve=session.data.getvalues(1:nr,2);
\end{lstlisting}

\color{lightgray} \begin{lstlisting}
ans = 

  <a href="matlab:help('IEEGDataset')">IEEGDataset</a>:

         snapName: 'I521_A0006_D001'
          montage: As Recorded
           filter: 'No Filter'
         resample: 'Not resampled'
       sampleRate: 2000
           values: 
    channelLabels: [2x2 cell]
         annLayer: []
      rawChannels: [1x2 IEEGTimeseries]
      allMontages: [1x44 IEEGMontage]

  <a href="matlab:methods(IEEGDataset)">Methods</a>, <a href="matlab:IEEGObject.openPortalSite()">main.ieeg.org</a>

\end{lstlisting} \color{black}
\begin{lstlisting}
durationInUSec = session.data(1).rawChannels(1).get_tsdetails.getDuration;
durationInSec = durationInUSec / 1e6;
\end{lstlisting}
\begin{lstlisting}
%set sample rate
sample_rate=2000;

%create filter
[b, a] = ellip(4,0.1,40, 300/(sample_rate/2),'high');
\end{lstlisting}
\begin{lstlisting}
%As computed by the code above, the denominator coefficients for my filter
%function are 1.000, -1.7432, 1.6167, -0.6559, 0.1430 the numerator coefficients for my filter function are
% 0.3420, -1.2740, 1.8676, -1.2740, 0.3420.
\end{lstlisting}

  \item Using the \verb|filter| function and \verb|filtfilt| function, obtain two different filtered outputs of the nerve signal.
      \begin{enumerate}
        \item In a 2x1 subplot, plot the first 50 ms of the unfiltered nerve signal in the top subplot; in the bottom subplot, plot the \verb|filter| output in blue and the \verb|filtfilt| output in red. Use a potential range (y-axis) of -20 to 50 millivolts. (4 pts)

\begin{lstlisting}
%create 50 ms time vector
time_ms=1/sample_rate:1/sample_rate:0.05;

time_ms=time_ms*1000;

%select 50 ms of nerve data
fifty_ms_nerve=(allData_nerve(1:100))/1000;
\end{lstlisting}
\begin{lstlisting}
%do the filtering using the ellip function output

%filter function
filter_out=filter(b,a, fifty_ms_nerve);

%filtfilt function
filt_filt_out=filtfilt(b,a,fifty_ms_nerve);
\end{lstlisting}
\begin{lstlisting}
%create subplot

figure
subplot(2,1,1)
plot(time_ms, fifty_ms_nerve)
ylim([-20, 50])
ylabel('Voltage (uV)')
xlabel('Time (milliseconds)')
title('Unfiltered Nerve Signal')
subplot(2,1,2)
plot(time_ms, filter_out,'b')
hold on
plot(time_ms, filt_filt_out,'r')
ylabel('Voltage (uV)')
xlabel('Time (milliseconds)')
title('Filtered Nerve Signal')
legend('Filter Output','FiltFilt Output')
ylim([-20, 50])
\end{lstlisting}


\includegraphics [width=5in]{clarkand_HW_6_01.png}

        \item How is the unfiltered signal different from the filtered signal? What is different about the two filtered (red and blue) signals? (2 pts)

\begin{lstlisting}
% The unfiltered signal is different from the filtered signal in that the points in the unfiltered signal's waveform have
% a higher amplitude (including much higher spikes) compared to the filtered signal. Additionally, the
% variation in the amplitudes of the points in the unfiltered wavefrom is
% higher than in the filtered waveform (in the filtered waveform the
% points are closer together to each other on the y-axis compared to the unfiltered waveform). The filtered
% (red and blue) signals are different in that the blue signal, generated
% using the MATLAB filter function, is phase shifted compared to both the
% red signal and the unfiltered data. However the red signal, generated
% with the MATLAB filtfilt function, is not phase shifted and the features
% of the signal in the red signal and the unfiltered data are at the same
% x-locations. Additionally, the red signal preserves the large positive voltage spike seen in the unfiltered
% data, while the blue signal cuts it down significantly. The amplitudes of the points in the red and blue signal are
% different throughout both waveforms (sometimes red is higher sometimes blue is higher). The blue waveform presents
% a significantly lower max voltage spike than the red waveform.






%generally have a higher amplitude compared to the point in the red signal's waveform,
% and the spikes in the red signal's waveform are usually higher.
%

%lower amplitude, shifted left to right (?)

%filt_filt output better
\end{lstlisting}

        \item Briefly explain the mathematical difference between the two filtering methods, and why one method might be more advantageous than the other in the context of spike detection? (5 pts)

\begin{lstlisting}
% The MATLAB filter function and filtfilt function differ in the way they
% filter input data. Specifically the MATLAB filter function represents a 1-D
% digital filter that uses a rational transfer function to filter the input
% data ("filter 1-D digital filter," 2020).
% However, the MATLAB filtfilt function does zero-phase digital filtering
% of the input data, and it filters the input data in the forward
% direction, reverses this now filtered data and puts it back through the
% filter ("filtfilt Zero phase digital filtering," 2020). Thus the filter created by the
% filtfilt function also has zero phase distortion, a transfer function equal to the squared magnitude of the original transfer function, and an order that is double
% the order specified by the transfer function numerator and demoniator
% coefficients inputted ("filtfilt Zero phase digital filtering,"2020). The MATLAB
% filtfilt method might
% be more advatangeous compared to the MATLAB filter method in the context of
% spike detection because the filtfilt method does have zero phase
% distortion, which means the filtfilt method will preserve the features of
% the unfiltered data exactly where they occur in the filtered waveform
% ("filtfilt Zero phase digital filtering,"2020). However, the filter function phase distorts the input data in the filtered output. This is
% important for spike detection because the time at which a spike occurs
% may yield insight on the behavior it is decribing, the cause of the spike, and/or affect its
% detection. As such, it would be best to use the filtfilt method so the input data
% is not phase-distorted when it is filtered in the context of spike detection tasks.

%References:
%filtfilt Zero phase digital filtering. (2020). Retrieved March 10, 2020, from https://www.mathworks.com/help/signal/ref/filtfilt.html

%filter 1-D digital filter. (2020). Retrieved March 10, 2020, from https://www.mathworks.com/help/matlab/ref/filter.html
\end{lstlisting}

      \end{enumerate}
        \item Using a spike threshold of +30 mV, calculate the index and value of the peak voltage for each spike in the \textbf{filtered} nerve signal (select the best one). Use these values to plot the first 2.5 seconds of the nerve signal with a red dot above (e.g. 10 mV above) each spike. (Hint: Plot the entire length of the nerve signal with all the spikes marked but then restrict the x-axis using \verb|xlim| to [0, 2.5] seconds) (4 pts)

\begin{lstlisting}
%run filt_filt on the whole dataset. filt_filt function is better than filter
%function

filt_filt_tot=filtfilt(b,a,(allData_nerve/1000));

%plotting 2.5 seconds of the filtered data

%create 2.5 second time vector
time_2=1/sample_rate:1/sample_rate:durationInSec+1/sample_rate;

%time_2=time_2*1000;

threshold = 30;

[pks, locs]=findpeaks(filt_filt_tot, time_2, 'MinPeakHeight', threshold)

%convert locatins to times in milliseconds
%loc_times=time_2(locs);
\end{lstlisting}

\color{lightgray} \begin{lstlisting}
pks =

   31.6925
   96.3585
  129.4620
   66.4497
   64.9381
   48.1245
  113.5636
   31.6831
   73.4865
  121.3998
   72.2594
   67.6410
   59.7600
   63.3552
   76.9529
  149.7190
   56.7418
   31.5806
   40.2886
   36.1789
   68.2765
   56.5004
  101.9634
  116.8954
   47.0143
   35.8406
   70.4530
   59.1729
   72.8805
   41.7954
   36.3824
   72.7740
   60.0195
   40.3204
  221.5279
   59.9390
   43.9736
   49.2995
   56.1479
   75.1737
   48.7109
   73.7691
   64.5988
   46.4497
   32.8721
   55.9038
   40.0355
  123.7623
   71.2277
   76.5734
   90.5888
  111.2634
  129.0722
   70.4437
   71.1535
   73.1953
   59.3883
   51.2822
   49.0915
   53.4192
   65.0858
  105.9240
   34.1493
  241.0532
   64.9728
   65.1605
   69.1698
   39.4916
   40.5063
  255.9609
   63.4450
   72.8219
   55.5940
   76.5320
   91.6765
   55.4257
   54.5481
   32.4706
   60.1194
   80.5031
   70.6279
   69.8016
   66.2803
   72.4268
   57.2744
  111.8370
  127.2979
   72.7814
  164.6414
   68.9846
   58.0100
   67.9810
  134.9118
   69.4285
   39.5566
   57.7727
   61.2482
   51.0701
   55.7471
   58.4129
   53.1688
   78.6903
   68.3587
   31.7614
  103.4938
  138.1865
  169.3965
   61.8389
   50.2495
   69.5257
   58.9046
  233.0956
   63.9710
  217.2062
   55.4727
   46.7552
   56.2481
   60.0178
  238.9544
   35.7451
   36.9681
   66.3855
   68.3506
  121.6303
   59.4667
  110.7238
   68.8822
   61.1741
   61.7399
   42.4897
   34.7618
   74.2428
   60.3789
  110.7981
  115.7268
   59.4329
   42.4143
   73.8505
   71.2924
   74.4465
   40.3112
   64.7095
   31.0649
  241.0039
   54.5879
   74.4801
   72.7829
   60.9047
   56.9205
   99.6857
  115.6780
   31.7564
  229.7175
   49.4638
   43.8003


locs =

  Columns 1 through 7

    0.0100    0.0685    0.0695    0.0890    0.0970    0.1460    0.1590

  Columns 8 through 14

    0.2075    0.2090    0.2670    0.2775    0.2990    0.3425    0.3925

  Columns 15 through 21

    0.3945    0.4200    0.4420    0.4695    0.4735    0.4745    0.5165

  Columns 22 through 28

    0.6015    0.6465    0.6475    0.6500    0.7065    0.7725    0.7825

  Columns 29 through 35

    0.8560    0.8605    0.8615    0.8920    0.9245    0.9260    0.9290

  Columns 36 through 42

    0.9845    0.9895    1.0525    1.0745    1.0755    1.0850    1.1735

  Columns 43 through 49

    1.1990    1.2600    1.2610    1.2645    1.2655    1.2755    1.2900

  Columns 50 through 56

    1.3070    1.3155    1.3580    1.3590    1.3740    1.3805    1.4365

  Columns 57 through 63

    1.4700    1.5110    1.5120    1.5130    1.5785    1.6175    1.6275

  Columns 64 through 70

    1.7015    1.7110    1.7120    1.7850    1.8480    1.8945    1.9055

  Columns 71 through 77

    1.9140    1.9300    1.9505    1.9755    1.9765    1.9930    2.0585

  Columns 78 through 84

    2.0600    2.1230    2.1705    2.1900    2.2030    2.2755    2.2770

  Columns 85 through 91

    2.3330    2.3395    2.3405    2.3535    2.3665    2.3860    2.4235

  Columns 92 through 98

    2.4630    2.5245    2.5700    2.5980    2.6615    2.6845    2.7755

  Columns 99 through 105

    2.7770    2.8215    2.8855    2.9035    2.9715    2.9775    2.9820

  Columns 106 through 112

    2.9830    3.0235    3.0335    3.0575    3.0960    3.1595    3.1780

  Columns 113 through 119

    3.2100    3.2695    3.3130    3.3545    3.4080    3.4135    3.4470

  Columns 120 through 126

    3.4840    3.5160    3.5510    3.6050    3.6200    3.6210    3.6350

  Columns 127 through 133

    3.6705    3.7110    3.7270    3.7950    3.8020    3.8500    3.9035

  Columns 134 through 140

    3.9135    3.9145    3.9245    3.9630    3.9700    3.9965    4.0280

  Columns 141 through 147

    4.0870    4.0920    4.1185    4.1695    4.1735    4.2000    4.2380

  Columns 148 through 154

    4.2950    4.3005    4.3235    4.3245    4.3630    4.3745    4.4080

  Column 155

    4.4305

\end{lstlisting} \color{black}
figure plot((allData\_nerve)/1000) figure plot(filt\_filt\_tot)
\begin{lstlisting}
%create plot
figure
plot(time_2,filt_filt_tot)
hold on
for i=1:length(locs)
    plot(locs(i),pks(i)+10, 'r.','MarkerSize',10)
    hold on
end
xlim([0 2.5])
ylabel('Voltage (mV)')
xlabel('Time (Seconds)')
title('Filtered Nerve Signal With Peaks Plotted')
legend('Filtered Nerve Signal', 'Peak', 'Location', 'northwest')
\end{lstlisting}


\includegraphics [width=5in]{clarkand_HW_6_02.png}
\begin{lstlisting}
% The filtfilt method was used to generated the plots above (the filtfilt method is the best one).
\end{lstlisting}

 \item Under the assumption that different cells produce different action potentials with distinct peak amplitudes, decide how many cells you think were recorded (some number between 1 and 6). You may find it helpful to zoom in and pan on the plot you made in question 1.3. You may also find it useful to plot the sorted peak values to gain insight into where ``plateaus'' might be. (No need to include these preliminary plots in the report, though.) Use thresholds (which you well set manually/by eye) to separate the different spikes. Make a plot of the first 2.5 seconds similar to that in 1.3 except now color the spike dots of each group a different color (e.g., \verb|'r.'|,\verb|'g.'|,\verb|'k.'|,\verb|'m.'|).(6 pts)

\begin{lstlisting}
%plot to visualize thresholds. no need to include it.
% figure
% yline(30)
% hold on
% yline(55)
% yline(100)
% yline(170)
% plot(time_2,filt_filt_tot)



threshold_1=30;
threshold_2=55;
threshold_3=100;
threshold_4=170;

[pks_1, locs_1]=findpeaks(filt_filt_tot, time_2, 'MinPeakHeight', threshold_1);
[pks_2, locs_2]=findpeaks(filt_filt_tot, time_2, 'MinPeakHeight', threshold_2);
[pks_3, locs_3]=findpeaks(filt_filt_tot, time_2, 'MinPeakHeight', threshold_3);
[pks_4, locs_4]=findpeaks(filt_filt_tot, time_2, 'MinPeakHeight', threshold_4);

figure
%Legend=cell(2,1);
for i=1:length(locs_1)
    h=plot(locs_1(i),pks_1(i)+10,'r.','MarkerSize',10);
    %Legend{1}=strcat('Cell 1 Peaks',num2str(1));
    hf(1)=h(1);
    hold on

end
for i=1:length(locs_2)
    k=plot(locs_2(i),pks_2(i)+10, 'g.','MarkerSize',10);
    %Legend{2}=strcat('Cell 2 Peaks',num2str(2));
    hf(2)=k(1);
    hold on
end
for i=1:length(locs_3)
    t=plot(locs_3(i),pks_3(i)+10, 'k.','MarkerSize',10);
    hf(3)=t(1);
    hold on
end
for i=1:length(locs_4)
    r=plot(locs_4(i),pks_4(i)+10, 'm.','MarkerSize',10);
    hf(4)=r(1);
    hold on
end

e=plot(time_2,filt_filt_tot);
hf(5)=e(1);
% %legend('Filtered Nerve Signal')
xlim([0 2.5])
ylabel('Voltage (mV)')
xlabel('Time (Seconds)')
title('Filtered Nerve Signal With Peaks Corresponding to the 4 Cells Plotted')
%legend(Legend)
%legend(p(1),'Cell 1 Peaks', h(1), 'Cell 2 Peaks')
legend(hf,{'Cell 1 Peaks','Cell 2 Peaks','Cell 3 Peaks','Cell 4 Peaks','Filtered Nerve Signal'}, 'Location', 'southoutside')
\end{lstlisting}


\includegraphics [width=5in]{clarkand_HW_6_03.png}

 \item Use Matlab's $k$-means\footnote{Clustering, like $k$-means you are using here, is a form of unsupervised learning.} function (\verb|kmeans|) to fit $k$ clusters (where $k$ is the number of cells you think the recording is picking up) to the 1D data for each spike.
  \begin{enumerate}
	\item Using the same color order (for increasing spike amplitude) as you did for the thresholds in question 1.4, plot the spike cluster colors as little dots slightly above those you made for question 1.4. The final figure should be a new plot of the nerve voltage and two dots above each spike, the first being your manual label and the second your clustered label, which (hopefully/usually) should be the same color. (4 pts)

\begin{lstlisting}
%concate all peaks into one vector
peaks_vector=[];

peaks_vector=vertcat(pks_1,pks_2, pks_3,pks_4);
\end{lstlisting}
\begin{lstlisting}
%run matlab's k-means clustering
rng(1)% set the seed
idx=kmeans(peaks_vector,4);% this will group all the data into four groups.

all_locs=vertcat(locs_1',locs_2',locs_3',locs_4');

%create a mtrix with the flag, location, and peak for plotting
k_means_mat=[idx all_locs peaks_vector ];
\end{lstlisting}
\begin{lstlisting}
%create plot
figure
%plot from 1.4
for i=1:length(locs_1)
    h=plot(locs_1(i),pks_1(i)+10,'r.','MarkerSize',10);
    %Legend{1}=strcat('Cell 1 Peaks',num2str(1));

    hf(1)=h(1);
    hold on

end
for i=1:length(locs_2)
    k=plot(locs_2(i),pks_2(i)+10, 'g.','MarkerSize',10);
    %Legend{2}=strcat('Cell 2 Peaks',num2str(2));
    hf(2)=k(1);
    hold on
end
for i=1:length(locs_3)
    t=plot(locs_3(i),pks_3(i)+10, 'k.','MarkerSize',10);
    hf(3)=t(1);
    hold on
end
for i=1:length(locs_4)
    r=plot(locs_4(i),pks_4(i)+10, 'm.','MarkerSize',10);
    hf(4)=r(1);
    hold on
end

%plot peaks from k means
for i=1:length(k_means_mat)
    if k_means_mat(i,1)==3
        plot(k_means_mat(i,2),k_means_mat(i,3)+30,'r.','MarkerSize',10);
        hold on
    elseif k_means_mat(i,1) ==4
        plot(k_means_mat(i,2),k_means_mat(i,3)+30, 'g.','MarkerSize',10);
        hold on
    elseif k_means_mat(i,1) == 1
        plot(k_means_mat(i,2),k_means_mat(i,3)+30, 'k.','MarkerSize',10);
        hold on
    elseif k_means_mat(i,1) ==2
        plot(k_means_mat(i,2),k_means_mat(i,3)+30, 'm.','MarkerSize',10);
        hold on
    end
end

e=plot(time_2,filt_filt_tot);
hf(5)=e(1);

xlim([0 2.5])
ylabel('Voltage (mV)')
xlabel('Time (Seconds)')
title({'Filtered Nerve Signal With Peaks Corresponding to the 4 Cells Plotted as Dots.', 'Kmeans Peaks Plotted Above Manual Threshold'})
legend(hf,{'Cell 1','Cell 2','Cell 3 Peaks','Cell 4','Filtered Nerve Signal'}, 'Location', 'southoutside')
\end{lstlisting}


\includegraphics [width=5in]{clarkand_HW_6_04.png}

	\item Which labeling, your manual ones or the ones learned by clustering) seem best, or do they both seem just as good? (Again, panning over the entire plot may be helpful.) (2 pts)

\begin{lstlisting}
% As can be seen from the dots in the plot generated in part 1.5a, the clustering
% labels
% are almost the same as the dots created by manual plotting most all of the points. We can
% see however there are some points where the two labels are different (for example at x = 2.05 seconds).
% While the clustering and manual labels are almost the same (and thus the performance of the two methods is almost the same), it appears that the manual labeling is better at doing the
% spike sorting (assigning the spikes to the correct cell) because the
% kmeans clustering unsupervised learning algorithim can at best provide an
% approximation of the sorting that the human eye can do. As such kmeans will
% not be as good as a human manually sorting the different spikes with
% thresholding. Additionally, manual sorting will not utilize any of the
% assumptions about the input data that are inherent in the k-means
% clustering algorithm, making it a more rigorous way to do labeling.
\end{lstlisting}

  \end{enumerate}
 \item In this question,  you will test the hypothesis that the muscle potential responses are really only due to spikes from a subset of the cells you have identified in the previous two questions. First, plot the first 2.5 seconds of the muscle fiber potential and compare it with that of the nerve. Observe the relationship between spikes and the muscle fiber response. (No need to include this plot and observation in your report.)
     Now, calculate the maximum muscle fiber potential change\footnote{max voltage - min voltage} in the 25 ms\footnote{Note that this 25 ms window is somewhat ad hoc and is just what seems reasonable by eye for this data. It implies no underlying physiological time scale or standard.} window after each spike (with the assumption that spikes without any/much effect on the muscle fiber potential do not directly innervate it).
  \begin{enumerate}
   \item Using the cell groups you either manually defined or found via $k$-means clustering (just specify which you're using) again with different colors, plot a colored point for each spike where the x-value is the spike amplitude and the y-value is the muscle potential change. (6 pts)

\begin{lstlisting}
%use the manual groups

%first plot 2.5 seconds of the muscle response and compare it with the
%nerve
% figure
% plot(time_2, allData_nerve/1000)
% xlim([0 2.5])
% figure
% plot(time_2, allData_muscle/1000)
% xlim([0 2.5])

mv_muscle_data=allData_muscle/1000;
%define all the peak times
all_peak_times=k_means_mat(:,2);

max_potential_change=[];%list of potential changes

%now calculate the maximum muscle fiber potential change (max voltage - min voltage) in the 25 ms
%window after each spike
for j=1:length(all_peak_times)
    %pull out 25ms = 0.025 seconds of muscle data. sample rate is 2000 Hz. so 0.025 sec of data
    %is 50 data points.
    index_1=find(time_2==all_peak_times(j));%define the first index
    voltage_window=mv_muscle_data(index_1:index_1+49);%define the voltage window
    max_potential_change(j)=max(voltage_window)-min(voltage_window);%calculate max potential change
end
%
% %create plot with the colors using the k means method
% figure
% for k=1:length(max_potential_change)
%     if k_means_mat(k,1)==3
%         plot(k_means_mat(k,3),max_potential_change(k),'r.','MarkerSize',10);
%         hold on
%     elseif k_means_mat(k,1) ==4
%         plot(k_means_mat(k,3),max_potential_change(k), 'g.','MarkerSize',10);
%         hold on
%     elseif k_means_mat(k,1) == 1
%         plot(k_means_mat(k,3),max_potential_change(k), 'k.','MarkerSize',10);
%         hold on
%     elseif k_means_mat(k,1) ==2
%         plot(k_means_mat(k,3),max_potential_change(k), 'm.','MarkerSize',10);
%         hold on
%     end
% end
% %plot(k_means_mat(:,3),max_potential_change,'.','MarkerSize',10)
% xlabel('Peak Amplitude (mV)')
% ylabel('Max Muscle Potential Change (mV)')
% title('Max Muscle Potential Change v. Peak Amplitude Sorted Using K-Means Algorithm')

%create plot with the colors using the manual sorting thresholding method
figure
for i=1:length(locs_1)
    h=plot(pks_1(i),max_potential_change(i),'r.','MarkerSize',10);
    %Legend{1}=strcat('Cell 1 Peaks',num2str(1));
    gf(1)=h(1);
    hold on

end
for i=1:length(locs_2)
    k=plot(pks_2(i),max_potential_change(i+155), 'g.','MarkerSize',10);
    %Legend{2}=strcat('Cell 2 Peaks',num2str(2));
    gf(2)=k(1);
    hold on
end
for i=1:length(locs_3)
    t=plot(pks_3(i),max_potential_change(i+268), 'k.','MarkerSize',10);
    gf(3)=t(1);
    hold on
end
for i=1:length(locs_4)
    r=plot(pks_4(i),max_potential_change(i+298), 'm.','MarkerSize',10);
    gf(4)=r(1);
    hold on
end
%plot(k_means_mat(:,3),max_potential_change,'.','MarkerSize',10)
xlabel('Peak Amplitude (mV)')
ylabel('Max Muscle Potential Change (mV)')
title('Max Muscle Potential Change v. Peak Amplitude Sorted Using Manually Defined Thresholds')
legend(gf,{'Cell 1','Cell 2','Cell 3','Cell 4'})
\end{lstlisting}


\includegraphics [width=5in]{clarkand_HW_6_05.png}

   \item Does this plot support the hypothesis that the muscle fiber responses are only due to a subset of the cells. Explain why or why not. (3 pts)

\begin{lstlisting}
% No, this plot does not support the hypothesis that the muscle fiber responses
% are only due to a subset of cells. This is because we can clearly see
% that each cell (as grouped by manual sorting) creates a
% meaningful change in muscle potential compared to each of the other respective cells.
% Specifically cell 1 creates a change in muscle potential up to 9.6 mV,
% cell 2 creates a change in muscle potential up to 9.5 mV, cell 3 creates a
% change in muscle potential up to 9.4 mV, and cell 4 creates a change in
% muscle potential up to 6.8 mV. Thus, there were 4 cells recorded in this
% experiment and all of them contributed to a relatively large change in
% muscle potential and as such this plot does not support the hypothesis
% that the muscle fiber repsonses are only due to a subset of cells. The plot supports the
% hypothesis that the muscle fiber responses are due to all cells.
\end{lstlisting}

  \end{enumerate}
\end{enumerate}
\section{Multivariate Clustering (21 pts)}
In this section, you will explore similar methods for spikes sorting and clustering but with a different dataset, the human intracranial data in \verb|I521_A0006_D002|,
which is a larger dataset of the same recording you saw in \verb|I521_A0001_D001| of Homework 1.
  \begin{enumerate}
   \item Using a threshold six standard deviations above the mean of the signal, detect the spikes in the signal. In addition, extract the waveform from 1 ms before the peak to 1 ms after it with peak value in the middle. (You will end up with a matrix where each row corresponds to the number of data points in 2 ms of signal minus 1 data point. Use the closest integer number of data points for the $\pm$ 1 ms window.)

\begin{lstlisting}
%create new session


session2 = IEEGSession('I521_A0006_D002', 'andrewc', '/Users/andrewclark/Downloads/ieeg_password.bin' );
\end{lstlisting}

\color{lightgray} \begin{lstlisting}IEEGSETUP: Adding 'ieeg-matlab.jar' to dynamic classpath
Warning: Objects of edu/upenn/cis/db/mefview/services/TimeSeriesDetails class
exist - not clearing java 
Warning: Objects of edu/upenn/cis/db/mefview/services/TimeSeriesInterface class
exist - not clearing java 
IEEGSETUP: Found log4j on Java classpath.
URL: https://www.ieeg.org/services
Client user: andrewc
Client password: ****
\end{lstlisting} \color{black}
\begin{lstlisting}
%session2.data;

sample_rate2=32258;
\end{lstlisting}
\begin{lstlisting}
nr2 = ceil((session2.data.rawChannels(1).get_tsdetails.getEndTime)/1e6*session2.data.sampleRate);
\end{lstlisting}
\begin{lstlisting}
allData= session2.data.getvalues(1:nr2,1);
\end{lstlisting}


\includegraphics [width=5in]{clarkand_HW_6_06.png}
\begin{lstlisting}
durationInUSec_2 = session2.data(1).rawChannels(1).get_tsdetails.getDuration;
durationInSec_2 = durationInUSec_2 / 1e6;%178.8449 seconds
\end{lstlisting}
\begin{lstlisting}
%figure
%plot(allData)
%allData_mv=allData/1000;



mean_data=mean(allData);
std_data=std(allData);
threshold_p2 = mean_data + 6*std_data;

%run find peaks function to detect spikes in the signal
[pks_2, locs_2]=findpeaks(allData);
locs_2=locs_2(pks_2>threshold_p2);
pks_2=pks_2(pks_2>threshold_p2);


%[pks_p2, locs_p2]=findpeaks(allData, 'MinPeakHeight', threshold_p2);

%sampling rate is 32258 Hz, so 1 ms of data is 1e-3* 32258= 32.2580, rounds to 32
%datapoints of data

%extract the waveform from 1ms before the peak to 1ms after peak
wave_form_mat=zeros(length(locs_2),65);

for u=1:length(locs_2)
    index_left=locs_2(u)-32;
    index_right=locs_2(u)+32;
    wave_form_mat(u,:)=allData(index_left:index_right);
    %index_1=find(time_2==all_peak_times(j));%define the first index
    %voltage_window=mv_muscle_data(index_1:index_1+49);%define the voltage
    %window
end
\end{lstlisting}

	\begin{enumerate}
	  \item Plot the waveforms of all the spikes overlaid on each other in the same color. (4 pts)

\begin{lstlisting}
%create 2.5 second time vector
%time_2=1/sample_rate:1/sample_rate:durationInSec+1/sample_rate;

time_plot=(1/65:(2/65):2);

%create plot
figure
for s=1:308
    plot(time_plot,wave_form_mat(s,:),'-b')
    hold on
end
xlabel('Time (ms)')
ylabel('Voltage (uV)')
title('Waveforms of All Spikes Overlaid On Each Other')
\end{lstlisting}


\includegraphics [width=5in]{clarkand_HW_6_07.png}

	  \item Does it looks like there is more than one type of spike? (1 pt)

\begin{lstlisting}
% Yes it looks like there are two different types of spikes, one type has a higher amplitude than the other.
\end{lstlisting}

	\end{enumerate}
   \item For each spike, represent the waveform by its  principal components. Use the \verb|pca| command in Matlab. Intuitively, principal component analysis finds the coordinate system that most reduces the variability in your data.
	\begin{enumerate}
	  \item Run principal component analysis on all the spike waveforms and represent your data with the top two principal components. Make a scatterplot of your data in this principal component (PC) space. (3 pts)

\begin{lstlisting}
% Run PCA on all spike waveforms and represent data with top two principal
% components.

transposed_mat=wave_form_mat;

[coeff,score,latent,~,explained]=pca(transposed_mat);

%Represent data in the PC space
%PC_space=transposed_mat*coeff(:,1:2);
\end{lstlisting}
\begin{lstlisting}
%create plot
figure
plot(score(:,1),score(:,2),'.','MarkerSize',14)
xlabel('Principal Component 1 (uV)')
ylabel('Principal Component 2 (uV)')
title('Top Two Principal Components of the Data')
% figure
% bar(latent)
\end{lstlisting}


\includegraphics [width=5in]{clarkand_HW_6_08.png}

	  \item Each PC also has an associated eigenvalue, representing the amount of variance explained by that PC. This an output of the \verb|PCA| command. Plot the  principal component vs the total (cumulative) percent variance explained. What is the percent variance explained if you include the top two principal components? (3 pts)

\begin{lstlisting}
explained_plotted=[];
for i=1:length(explained)
    value=explained(1:i);
    explained_plotted(i)=sum(value);
end

figure
%plot(explained,score,'b.','MarkerSize',10)
plot(explained_plotted)
xlabel('Number of Principal Components')
ylabel('Percent Variance Explained (%)')
top_two_explained=explained(1)+explained(2);
title('Principal Component v. Total Percent Variance Explained')
ylim([60 105])
\end{lstlisting}


\includegraphics [width=5in]{clarkand_HW_6_09.png}
\begin{lstlisting}
% The top two principal components explain 73.7349% of the total variance.
\end{lstlisting}

	  \item Does it look like there is more than one cluster of spikes? (1 pt)
	\end{enumerate}

\begin{lstlisting}
%Yes, it looks like there are more than one cluster of spikes. It looks like there are two clusters of spikes.
\end{lstlisting}

   \item Use the same \verb|kmeans| function as you used before to cluster the spikes based on these two (normalized) features (the waveforms represented by the top two PCs). You will use a slight twist, though, in that you will perform $k$-medians (which uses the medians instead of the mean for the cluster centers) by using the \verb|'cityblock'| distance metric (instead of the default \verb|'sqEuclidean'| distance). Make a plot similar to that in 2.2.a but now coloring the two clusters red and green. (3 pts)

\begin{lstlisting}
%run k-medians

idx_2=kmeans(score(:,1:2),2,'Distance','cityblock');

%create plot
figure
for i=1:length(score)
    if idx_2(i)==1
        q=plot(score(i,1),score(i,2),'r.','MarkerSize',14);
        qf(1)=q;
        hold on
    elseif idx_2(i)==2
        o=plot(score(i,1),score(i,2),'g.','MarkerSize',14);
        qf(2)=o;
        hold on
    end
end
xlabel('Principal Component 1 (uV)')
ylabel('Principal Component 2 (uV)')
title('Top Two Principal Components of the Spike Data Sorted by K-Medians')
legend(qf,{'Spike Cluster 1','Spike Cluster 2'})
\end{lstlisting}


\includegraphics [width=5in]{clarkand_HW_6_10.png}

  \item Make a plot similar to 2.1 but now coloring the traces red and green according to which cluster they are in. Overlay the mean of the waveforms in each cluster with a thick black line (use the parameter \verb|'LineWidth'| and value \verb|'4'|). (3 pts)

\begin{lstlisting}
%create plot with colored traces.

%figure out how many 1s and 2s there are in the idx_2
count_1s=0;
count_2s=0;
for i=1:length(idx_2)
    if idx_2(i)==1
        count_1s=count_1s+1;
    elseif idx_2(i)==2
        count_2s=count_2s+1;
    end

end
\end{lstlisting}
\begin{lstlisting}
figure
cluster_1=zeros(count_1s,65);
cluster_2=zeros(count_2s,65);
for s=1:308
    if idx_2(s)==1
        e=plot(time_plot,wave_form_mat(s,:),'-r');
        cluster_1(s,:)=wave_form_mat(s,:);
        kl(1)=e;
        hold on
    elseif idx_2(s)==2
        z=plot(time_plot,wave_form_mat(s,:),'-g');
        %cluster_2=[cluster_2 wave_form_mat(s,:)];
        cluster_2(s,:)=wave_form_mat(s,:);
        kl(2)=z;
        hold on
    end
end

cluster_1_filtered=cluster_1(any(cluster_1,2),:);
cluster_2_filtered=cluster_2(any(cluster_2,2),:);
mean_cluster_1=mean(cluster_1_filtered,1);
hold on
mean_cluster_2=mean(cluster_2_filtered,1);
hold on
q=plot(time_plot,mean_cluster_1,'k-','LineWidth',4);
kl(3)=q;
hold on
plot(time_plot,mean_cluster_2,'k-','LineWidth',4)
xlabel('Time (ms)')
ylabel('Voltage (uV)')
title('Waveforms of All Spikes Overlaid On Each Other With Means Of Each Shown')
legend(kl,{'Spike Cluster 1','Spike Cluster 2','Means Of Each Cluster'})
\end{lstlisting}


\includegraphics [width=5in]{clarkand_HW_6_11.png}

  \item What are some dangers of using the clustering techniques in this homework? (List 3) (3 pts)

\begin{lstlisting}
% Some of the dangers of using the clustering techniques in this homework
% include: First, k-means clustering assumes that each cluster present in
% the input data should have roughly the same about of datapoints in it
% (Robinson, 2015). This is a danger because if the clusters actually have different numbers of
% datapoints the k-means clustering unsupervised learning algorithm may
% group the data non-optimally, leading to inaccurate
% conclusions. Second, the behavior of the k-means clustering algorithm is also
% greatly effected by the presence of outliers in the input data, when
% outliers are present the output clusters may be created non-optimally
% leading to a poor representation of the data (Franklin, 2019).
% A third danger that the input data to the k-means clustering technique used in this
% homework should actually have some inherehent clusters in it or else
% k-means clustering will not applicable; that is if k-means clustering is
% run on a uniform data (non-clustered) dataset the k-means clustering algorithm will partition
% the data into clusters, even though the clusters would be meaningless in
% this case (Kim, 2015).



%References:
%Franklin, J. (2019, November 16). Effect of outliers on K-Means algorithm using Python. Retrieved March 10, 2020, from https://medium.com/analytics-vidhya/effect-of-outliers-on-k-means-algorithm-using-python-7ba85821ea23

%Kim, K. (2015, February). How to understand the drawbacks of K-means. Retrieved March 10, 2020, from https://stats.stackexchange.com/questions/133656/how-to-understand-the-drawbacks-of-k-means

%Robinson, D. (2015, January 16). K-means clustering is not a free lunch . Retrieved March 10, 2020, from http://varianceexplained.org/r/kmeans-free-lunch/
\end{lstlisting}

\end{enumerate}
\end{document}




\end{document}
    
